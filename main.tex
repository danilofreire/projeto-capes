\documentclass[a4paper,11pt]{article}
\usepackage[utf8]{inputenc}
\usepackage[T1]{fontenc}
\usepackage{helvet} % Helvetica
\renewcommand{\familydefault}{\sfdefault}
\usepackage{lmodern}
\usepackage{color}
\definecolor{darkblue}{rgb}{0.0, 0.0, 0.55}
\usepackage{setspace}
\usepackage[left=2cm, right=2cm, top=2cm, bottom=2cm]{geometry}
\usepackage[backref,pagebackref]{hyperref}
\usepackage[authoryear]{natbib}
\usepackage{graphicx}
\usepackage{lscape}
\usepackage{indentfirst}
\usepackage{tikz}
\usepackage[brazilian]{babel}
\usepackage{babelbib} 
\newcommand\bookepigraph[4]{
\vspace{1em}\hfill{}\begin{minipage}{#1}{\begin{spacing}{0.9}
\small\noindent\textit{#2}\end{spacing}
\vspace{2em}
\hfill{}{#3}\small\\
\vspace{-2em}\begin{flushright}{#4}\end{flushright}}\vspace{1em}
\end{minipage}}
\exhyphenpenalty=1000
\hyphenpenalty=1000
\widowpenalty=1000
\clubpenalty=1000

\renewcommand*{\backref}[1]{}
\renewcommand*{\backrefalt}[4]{%
    \ifcase #1 (N\~{a}o citado.)%
    \or        Citado na p\'{a}gina~#2.%
    \else      Citado nas p\'{a}ginas~#2.%
    \fi}
\renewcommand{\backreftwosep}{ e~} 
\renewcommand{\backreflastsep}{ e~}

\hypersetup{pdftitle={A Ordem Violenta: Poder e Governança no Primeiro Comando da Capital},pdfauthor={Danilo Alves Mendes Freire},pdfsubject={Projeto de Pesquisa de Doutorado},pdfkeywords={Gangues prisionais, organizações criminosas, PCC, São Paulo},linkcolor=darkblue, citecolor=darkblue, urlcolor=darkblue, breaklinks=true, colorlinks=true}

%opening

\title{A Ordem Violenta: Poder e Governança no\\ Primeiro Comando da Capital}
\author{Danilo Alves Mendes Freire}
\date{19 de Janeiro de 2015}

\begin{document}
\maketitle

\begin{abstract}
\onehalfspacing

O presente projeto de pesquisa visa analisar como o Primeiro Comando da Capital (PCC), facção criminosa sediada no estado de São Paulo, desenvolveu mecanismos de governança privada para resolver seus problemas de ação coletiva. Especificamente, os objetivos principais da tese são: 1) elucidar sob quais condições a facção fez uso de diferentes instrumentos -- violência seletiva ou indiscriminada, incentivos monetários ou políticos -- para manter a colaboração e punir desertores durante sua história; 2) formular uma teoria de governança não-institucional a partir das práticas do PCC, enfatizando como o grupo criou, com sucesso, uma ordem social em um cenário de grandes restrições individuais. Nesse sentido, a tese dialoga não apenas com a produção nacional e estrangeira sobre gangues carcerárias, mas também com a extensa literatura de ação coletiva, formação do estado e imposição da ordem por meios coercivos. Como marcos teóricos, o trabalho adotará conceitos da teoria de sinalização, escolha pública e sociologia analítica, todas abordagens ainda inéditas nos estudos de violência no Brasil. Incentivos criados pela estrutura dos mercados ilícitos, problemas de agente-principal e sinalização sob condições de incerteza são considerados conceitos-chave para entendermos a estrutura do PCC. No que tange à metodologia, a tese empregará métodos mistos para cumprir seus objetivos. Enquanto fontes qualitativas, tais como entrevistas em profundidade, fornecerão os elementos factuais mais importantes para desenvolvermos os argumentos propostos, análises estatísticas e modelagem formal baseada em agentes serão utilizadas para generalizarmos os achados da pesquisa. Por fim, o trabalho discutirá aspectos relevantes das recentes ações governamentais e oferecerá subsídios para a formulação de políticas de segurança pública mais inclusivas. Com efeito, uma análise detalhada das organizações prisionais, tópico da tese em questão, irá não somente ampliar o presente conhecimento acerca da emergência de ordem em cenários decentralizados, como também irá colaborar, em termos práticos, para uma melhor atuação do poder público no combate ao crime e à violência urbana.\\

\noindent
\textsc{Palavras-Chave}: Ação coletiva, gangues prisionais, organizações criminosas, PCC, São Paulo
\end{abstract}

\newpage

\bookepigraph{3.75in}{O que é mais perigoso a respeito da violência é sua racionalidade. Claro que a violência por si só é terrível, mas a mais profunda raiz da violência, e também de sua permanência, deriva do tipo de racionalidade que utilizamos. A idéia tem sido a de que, se vivêssemos em um mundo da razão, estaríamos livres da violência. Isso é falso. Entre violência e racionalidade não há incompatibilidade. Minha questão não é colocar a razão sob julgamento, mas saber por que tal racionalidade é tão compatível com a violência.}{Michel Foucault}{\textit{A Verdade Está no Futuro}, p. 299.}

\section{Introdução}
\onehalfspacing

Como as organizações autônomas mantêm sua ordem interna sem recorrer às instituições governamentais? Pesquisas anteriores mostram que a oferta privada de bens públicos não só é teoricamente possível \citep{bergstrom1986private, hayek1960constitution, hayek1988fatal, olson1965logic}, como vem sido posta em prática por uma vasta gama de atores e grupos sociais \citep{baldassarri2009collective}. De empresas multinacionais \citep[25]{ostrom1990governing} a produtores agrícolas \citep{schepel2005constitution}, de partidos políticos \citep[729]{helmke2003informal} a senhores feudais chineses \citep{jackson2003warlords}, diversas instituições têm desenvolvido práticas de governança privada e criado instrumentos para fazer valer suas regras internas.

As organizações criminosas, por sua vez, enfrentam sérios problemas para estabelecer mecanismos informais de governança e solucionar seus problemas de ação coletiva \citep{sheptycki2003governance}. Face ao constante medo de serem detidos pela polícia \citep{williams2002cooperation}, os grupos criminosos encontram uma série de dificuldades tanto para negociar os produtos que comercializam, quanto para identificar seus pares e estabelecer canais duradouros de comunicação \citep[xii]{gambetta2009codes}. Ademais, como resultado da natureza violenta de suas atividades, os níveis de confiança entre criminosos são baixos e as relações interpessoais são sabidamente instáveis \citep{liebling2012social, von2004organized}. A cooperação entre criminosos, com efeito, tende a ser inconsistente e precária \citep{morselli2007efficiency}.

Dentro das prisões, todos esses problemas são levados às últimas consequências \citep{clemmer1940prison, sykes1960inmate}. As contínuas interações entre os detentos -- resultado natural do confinamento coletivo -- fazem com que os presos estejam frequentemente sob risco de perderem suas posses e serem agredidos pelos demais \citep{skarbek2011governance}. Além disso, são comuns os abusos cometidos pelas autoridades carcerárias, muitas vezes brutais e inesperados \citep{adorno1998prisoes, assis2008realidade, toch1977police}. Assim, em um cenário onde os indivíduos tendem a ser hostis e o custo das punições é frequentemente a morte, a teoria sugere que não há qualquer incentivo imediato para que um criminoso seja altruísta no longo prazo e busque uma saída benéfica para os demais. Os presos têm forte estímulo para serem \textit{free riders} e não arcarem com qualquer custo pessoal, ainda que isso resulte em um ambiente de \textit{bellum omnium contra omnes} \citep[16]{hobbes1983cive}. Desse modo, as gangues carcerárias são o caso extremo do que a teoria de ação coletiva define como \textit{dilema social}, isto é, uma situação na qual ações que são individualmente racionais levam a resultados que são coletivamente irracionais \citep[250]{heckathorn1996dynamics}. 

Diferente do que sugere o senso comum e as teses de ação coletiva, a experiência mostra que a cooperação entre presos é mais frequente do que se imagina. A gangues carcerárias nos Estados Unidos e na África do Sul, por exemplo, operam com grande eficiência há décadas, e mesmo com severas punições individuais, contam hoje com milhares de membros associados \citep{lotter1988prison, haysom1981towards}. Entretanto, se por um lado é sabido que a mera presença de interesse gerais não configura uma condição suficiente para resolver problemas coletivos \citep{marx2012achtzehnte,  olson1965logic}, por outro, os mecanismos pelos quais os grupos criminosos estabelecem suas regras ainda são ausentes da grande maioria dos trabalhos empíricos \citep{decker2007understanding}. \citet[2]{fleisher2001overview}, por exemplo, afirmam que ``[\dots] as gangues de prisão são a fronteira final dos pesquisadores e o pior pesadelo dos administradores penitenciários'', \citet[89]{simon2000thesociety} argumenta que as pesquisas sobre gangues prisionais virtualmente cessaram nos anos 1980, e \citet[10]{skarbek2014social} atesta que apenas eventos extraordinários como revoltas e motins têm despertado o interesse dos analistas sociais. Dessa forma, as organizações carcerárias são um objeto de pesquisa ainda inexplorado, embora claramente relevante para os interessados nas formas de cooperação sob incerteza, no papel comunicativo da violência, e nos incentivos utilizados por esses grupos para manter sua coesão no longo prazo \citep{campana2013cooperation, densley2012street, freeman1994crime}.

\subsection{Justificativa do Projeto}

O Primeiro Comando da Capital (PCC), organização criminosa oriunda dos presídios de São Paulo, é um excelente estudo de caso para avaliar as questões sugeridas acima. Pretende-se analisar como o PCC conseguiu resolver seus problemas de ação coletiva até os dias atuais. Em pouco mais de 20 anos, o Comando estabeleceu-se como a facção mais importante do País e exerce controle sobre mais de 90\% das prisões em seu estado de origem \citep{biondi2008etica}. Conforme relatório recente do Ministério Público Estadual de São Paulo, a facção consolidou sua presença em 22 dos 27 estados brasileiros, no Paraguai e na Bolívia, fatura cerca de 120 bilhões de reais por ano, e, alega-se, investiu parte dessa fortuna para eleger, indiretamente, candidatos favoráveis ao grupo \citep{veja2013}. Ademais, o PCC também articulou, em 2001 e 2006, as duas maiores rebeliões prisionais que se tem notícia no Brasil, e foi a única organização a monopolizar com sucesso o uso da violência no sistema carcerário paulista  \citep{dias2009guerra}. Sua importância como ator público, portanto, é inegável.

A pesquisa trará avanços consideráveis à produção nacional sobre as gangues prisionais, a saber: 1) a utilização de um arcabouço teórico consistente com as novas teorias de organização social e ação coletiva, já encontrado nas interpretações de gangues prisionais no exterior mas ainda inexistente na comunidade nacional, e 2) o uso das ferramentas teóricas da modelagem baseada em agentes e da sociologia analítica, que devido a seu foco nos \textit{mecanismos causais} permitem relacionar, de modo rigoroso, os macro-fenômenos às motivações individuais \citep{hedstrom1998social}. Tal ligação macro-micro é ainda escassa na produção acadêmica nacional, e aqui visamos preencher esta lacuna.

Os textos brasileiros, em sua vasta maioria escritos por sociólogos e juristas, são inegavelmente ricos em material histórico, porém permanecem tímidos em
ambições teóricas. Dentre os textos mais citados da bibliografia especializada, \textit{nenhum deles} buscou formular um arcabouço teórico substantivo a partir das experiências do PCC (ou das facções do Rio de Janeiro) que possa contribuir, por exemplo, para um melhor entendimento das gangues prisionais \textit{em geral} ou que expanda as teorias de decisão coletiva \citep[365]{dias2011pulverizaccao}. Atualmente, não há um diálogo construtivo entre a produção nacional e a literatura desenvolvida no exterior, e infelizmente a rica e variada coleção de dados sobre as gangues brasileiras ainda não fora devidamente mobilizada em um esforço de \textit{theory building} que poderia enriquecer o debate especializado em termos mais amplos. 

O trabalho em tela também terá relevância imediata para as discussões sobre políticas públicas no País. Ao discutir os mecanismos de ação coletiva utilizados do PCC, forneceremos subsídios para que o estado compreenda -- e, esperamos, reduza -- o domínio da facção criminosa dentro do sistema carcerário. Estabelecer o monopólio da força legítima está na própria definição do estado \citep{weber1919politik}, e a atual situação brasileira mostra ser necessário atentar para esta questão. Como destacado por \citet{adorno2002exclusao}, a persistência de usos privados da força no Brasil, seja por meio de execuções sumárias ou pelos linchamentos populares, deslegitimam a democracia perante os cidadãos e facilitam a adesão dos indivíduos a formas paralelas do poder, tais quais as facções criminais e os grupos milicianos. Dessa forma, desarticular o PCC colaboraria para enraizar o sentimento de ordem e segurança na população, e, por conseguinte, torná-la mais resiliente às formas extralegais de solução de conflitos.

Por fim, a pesquisa lançará luz sobre o modo pelo qual o governo tem sistematicamente tratado a população carcerária. Ao registrar e organizar as narrativas dos detentos, será possível traçar uma imagem vívida do estado precário do sistema prisional paulista. Conforme descrito por \citet[323]{dias2009guerra}, o recente processo de democratização do País, iniciado em 1985, pode apenas ser considerado bem-sucedido caso os atores estatais cumpram sua promessa de manter a ordem pública enquanto preservam os direitos civis dos indivíduos, e a prisão é claramente o caso mais emblemático desse desajuste. Dessa maneira, a tese colaboraria para uma discussão a respeito de como o estado, oferecendo dados empíricos que corroboram a ideia de que uma política penitenciária mais humana reduziria o valor dos bens públicos ofertados pelo PCC (segurança física e direitos de propriedade), faria o estado cumprir seu papel de provedor de direitos humanos e, por fim, facilitaria a futura ressocialização do preso.

\section{Objetivos}

\subsection{Objetivos Gerais}

O presente projeto de pesquisa possui duas metas a serem atingidas. A primeira delas é elucidar sob quais condições o PCC fez uso de distintas formas de incentivo para manter a colaboração e punir eventuais traidores. Pretendemos detalhar a relação entre os fatores estruturais -- que determinaram as oportunidades -- e escolhas particulares dos indivíduos, que por sua vez formam novos processos estruturais. Em 2002, a organização decidiu incorporar a \textit{igualdade} em seu lema (``paz, justiça e liberdade''), e dissolveu parcialmente a estrutura hierárquica que prevalecia de 1993 até então \citep{biondi2010junto}. Tal condição está em franco contraste com o que acontece em outras gangues descritas na literatura contemporânea \citep{leeson2010criminal}. Assim, o PCC mostra-se como um caso único para avaliar os diferentes incentivos (coercitivos ou não) no decorrer de sua história (1993--2001/2001--2006/2006--presente) \citep[165-181]{dias2011pulverizaccao}, cujas rupturas podem descrevem interessantes pontos de inflexão nas práticas adotadas pela facção.

Em segundo lugar, almeja-se formular uma teoria de governança não-institucional a partir das práticas da facção criminosa, destacando como os criminosos conseguiram manter uma ordem social mesmo em um cenário de incentivos adversos. Embora haja uma literatura relativamente vasta sobre a cultura prisional, teorias que tratam dos \textit{mecanismos de governança} desses grupos são praticamente inexistentes \citep{skarbek2011governance, skarbek2014social}. Atualmente, sabe-se bem pouco sobre como as organizações formadas por presidiários exercitam seu poder, tanto no curto quanto no longo prazo \citep{roberts2014prison}. Como resultado desse desconhecimento, as políticas penitenciárias são ainda notadamente ineficientes no combate à violência prisional, e tendem a continuar assim caso não haja uma mobilização no sentido de compreender as gangues dos presídios.

Tais objetivos serão desdobrados em hipóteses concretas e empiricamente testáveis, como os apresentados abaixo.

\subsection{Objetivos Específicos}

A fim de avaliar como o PCC lança mão de diferentes formas de incentivo para resolver seus problemas coletivos, meu objetivo é testar o seguinte conjunto de hipóteses:\\

$H_1$: \textit{O aumento da população carcerária causa pressão no direito de propriedade dos presos, o que gera incentivos para que os detentos organizem uma facção central}\\

O crescimento do número de presos em São Paulo nas últimas décadas e notório. O estado hoje possui cerca de 200,000 detentos -- 36\% do total do País -- e encarcera cerca de 15,000 novos presos por ano \citep{ig2014numeropresos}. Na primeira hipótese deste trabalho argumento que este foi o motivo principal que levou à demanda por um órgão de governança central dos detentos. A pressão populacional no já precário sistema carcerário no estado levou a graves tensões entre os presos, uma vez que a disputa por recursos escassos (comida, espaço, drogas, armamentos brancos) torna-se cada vez mais acirrada em um cenário de crescimento demográfico \citep[275]{darke2013inmate}. Assim, seguindo a literatura de escolha pública (ver seção 4.1), que lida com incentivos para proteção privada sob condições de anarquia \citep{buchanan1975limits, nozick1974anarchy, olson1965logic}, argumento que como cada preso detinha progressivamente menos poder \textit{de facto} sobre sua propriedade, tal insegurança gerou incentivos para que se formasse uma estrutura monopolista de violência dentro do sistema carcerário. Como sugere \citet{buchanan1973defense}, um monopólio da violência é melhor do que a oferta de proteção em um mercado competitivo, uma vez que o monopolista tem um incentivo para ``subproduzir'' violência e alocar tais recursos em outros lugares. Em teoria, pode-se dizer que a ``subprodução'' da violência é Pareto-superior em um cenário de equilíbrio parcial.\\

$H_2$: \textit{O uso de violência letal indiscriminada era a principal ferramenta de governança do PCC durante seu primeiro estágio de consolidação}\\

Seguindo a vasta linha de trabalhos sobre a formação dos estados via coerção \citep{bates1981markets, herbst2000states, tilly1992coercion}, esta hipótese argumenta que, na primeira fase de existência do PCC (1993--2001), o instrumento escolhido pelo grupo era a coerção física, sobretudo a violência letal. Primeiramente, podemos adaptar o argumento originalmente apresentado por \citet{tilly1992coercion} para explicar a criação dos estados europeus e utilizá-lo para entender a formação do PCC. O mecanismo causal que sugiro é que assim que os presos perceberam a demanda por proteção de seus pares, inicia-se uma disputa violenta para entender \textit{qual grupo} seria o provedor de segurança privada entre os presos \citep{gambetta1996sicilian, skarbek2011governance}. Nesse estágio, o PCC não possuía informação suficiente para empregar violência seletiva nos presídios e não poderia antecipar, no futuro, quais seriam seus futuros membros \citep{kalyvas2006logic}, o que torna a violência indiscriminada a única alternativa para a facção. Com o número de filiados ainda relativamente pequeno, era difícil para o grupo obter informações confiáveis e antecipadas a respeito do comportamento da maioria dos presos. Uma vez que os demais detentos tinham dúvidas a respeito do tamanho e força do Comando, havia incentivos para que estes permanecerem à margem do PCC. Nesse sentido, o PCC precisava mostrar rapidamente que suas ameaças são críveis, e a força letal tornou-se a estratégia dominante nesse tipo de cenário também graças a seu papel \textit{comunicativo}. Como prevê a teoria de sinalização\footnote{Uma discussão detalhada sobre a teoria pode ser vista na seção 4.1 deste projeto.}, o uso de violência indiscriminada tinha o propósito de comunicar, com um sinal inequívoco, que a facção estava determinada a monopolizar a violência carcerária.\\

$H_3$: \textit{A distribuição de incentivos monetários e políticos dentro do PCC é resultado do uso bem-sucedido da violência indiscriminada nos primeiros estágios}\\

Aqui sugiro que a diversificação das táticas de governança -- que inclui, por exemplo, a ajuda financeira às família dos detentos \citep{camara2005} -- só se tornou viável depois da criação de uma base que pudesse ser ``taxada'' na prisão (2001--presente). Este argumento está ligado à teoria do ``bandido estacionário''\footnote{Todas as traduções do inglês, francês e alemão presentes neste projeto são livres e de minha autoria.} (\textit{stationary bandit}) proposta por \cite{olson1993dictatorship} e revisitada por \cite{skarbek2011governance}: o PCC tem um forte incentivo para investir em melhorias para os presos e suas famílias, uma vez que ele espera extrair renda a partir dos negócios exercidos, dentro ou fora da prisão, por seus associados.

Assumo que, nas fases posteriores da organização do PCC, o maior problema enfrentado pelo grupo é o dos \textit{custos de coordenação}. Como definido no clássico texto de \citet[303]{becker1994division}, conforme uma instituição cresce e diversifica suas atividades, há um aumento no número de problemas de agente-principal, de oferta e de comunicação interna, o que reduz a eficiência do grupo. Na hipótese acima, argumento que estratégia que o PCC adotou para resolver esse dilema foi a de introduzir outras formas de incentivo para seus membros após a conquista do monopólio territorial dos presídios, sendo a mais importante delas a distribuição de incentivos monetários e a semi-dissolução da estrutura hierárquica que regia a instituição. Com a presente hipótese, argumento que o PCC atravessou o mesmo processo e que a capacidade para o grupo se transformar em ``bandido estacionário'' veio da liberação de recursos, financeiros e humanos, obtidos com o monopólio da violência.\\

$H_4$: \textit{Presídios nas quais o PCC possui controle incompleto tendem a ter mais violência entre detentos e abusos da administração penitenciária}\\

A última hipótese sugere que em áreas sobre os quais o PCC possui controle parcial há maiores oportunidades para os presos aliarem-se a outros grupos ou seguirem como ``independentes'' na detenção. Com isso, o risco de fragmentação do poder gera mais oportunidades para enfrentamentos diretos em um determinado presídio, o que tende a gerar maiores níveis de violência no lugar. De modo complementar, há também o problema de informação incompleta, assinalado por \citep{kalyvas2006logic} e mencionado na primeira hipótese deste trabalho. Como não é viável ao PCC utilizar violência seletiva, resta-lhe utilizar violência indiscriminada, o que gera uma série de danos colaterais para o grupo. Tais situações levam a um aumento no número de possíveis conflitos no mesmo território, e muitas dessas interações tornam-se violentas.

Há também outro resultado derivado dessa situação de domínio imperfeito, ainda não discutida detalhadamente pela literatura especializada. Abusos policiais contra a população carcerária são um fenômeno recorrente nas prisões de São Paulo \citep{adorno1998prisoes, assis2008realidade}. Um dos serviços mais importantes do PCC é justamente a proteção dos detentos contra abuso físico \citep{dias2011pulverizaccao}, e por conseguinte sugiro a hipótese de que caso o PCC controle uma determinada prisão, ele pode deter comportamentos agressivos por parte da administração carcerária com ameaças críveis de retaliação ou revolta. O PCC, portanto, é capaz de obter uma redução da violência institucional contra os presos em troca de um ambiente de trabalho mais pacífico para os funcionários do presídio e, com isso, obter maior legitimidade no longo prazo perante presos e os agentes. 

\section{A Literatura sobre o Primeiro Comando da Capital}

Há décadas os pesquisadores têm se debruçado sobre o tema da cultura prisional \citep[398]{hunt1993changes}. Desde os pioneiros trabalhos de \cite{clemmer1940prison}, \cite{sykes1960inmate}, \cite{goffman1961characteristics}, \cite{berk1966organizational} e \cite{tittle1969inmate}, é significativa a produção intelectual acerca das interações entre detentos e seus modos de organização social. Em sentido amplo, as gangues prisionais podem ser definidas como ``[...] organizações que operam dentro do sistema carcerário como entidades orientadas para o crime, consistindo em um seleto grupo de prisioneiros que estabeleceram uma cadeia organizada de comando e que são governados por um código de conduta pré-estabelecido'' (\citeauthor{lyman1989gangland}, \citeyear{lyman1989gangland}, 89 apud \citeauthor{delisi2004gang}, \citeyear{delisi2004gang}, 371). No que tange à produção acadêmica nacional, \citet[365]{dias2011pulverizaccao} assinala que embora um grande contingente de autores ainda esteja ligado à dimensão estritamente jurídica do fenômeno do crime organizado, a literatura que descreve as facções criminosas vem crescendo substancialmente nas últimas décadas. Além dos relatos de cunho jornalístico que descrevem o surgimento e expansão das atividades de grupos criminosos \citep{amorim2003cv, barcellos2003abusado, jozino2004cobras, souza2007pcc}, há também trabalhos específicos sobre o tráfico de drogas no Rio de Janeiro \citep{lessing2008facccoes, zaluar1999debate}, sobre as revoltas nas prisões de São Paulo \citep{adorno2007organized,da2009crime, salla2006rebelioes}, o trabalho da inteligência na atuação contra o crime \citep{mingardi2007trabalho}, as relações entre crime organizado e o sistema prisional \citep{ramalho1979mundo, pieta1993pavilhao, porto2007crime} e, o que é mais relevante para o presente estudo, sobre as origens e funcionamento do PCC no sistema carcerário paulista \citep{biondi2010junto, dias2009guerra, dias2011pulverizaccao, marques2010liderancca}. 

\citet{thompson1980questao} é talvez o primeiro autor moderno a tratar da ``questão penal'' no Brasil. Fortemente influenciado pela idéia de ``prisionização'' (\textit{prisonization}) formulada por \citet{clemmer1940prison}\footnote{O processo de ``prisionização'' mencionado por \citeauthor{clemmer1940prison} descreve o fato de que os detentos progressivamente adotam a cultura das prisões, ela mesma criada a fim de resolver problemas oriundos do encarceramento. \cite{sykes1958society} expande esse conceito e argumenta que, embora ela não seja totalizante como descrita por \citeauthor{clemmer1940prison}, tal processo ocorre independente do tipo ou localização das prisões (nos EUA), e que ``[...] este sistema de valores toma a forma de um código dos detentos, que é usado como um guia para o comportamento dos presos com seus pares e os guardas. Portanto, o código dos detentos resume as expectativas comportamentais dos detentos em seu sistema social \citep[429]{paterline1999structural}.}, \citeauthor{thompson1980questao} indica que a ``cultura das prisões'' é o que permite aos indivíduos sobreviver perante dificuldades do encarceramento. Contudo, ao mesmo tempo que tais hábitos e costumes prisionais são interiorizados, eles dificultam o retorno do indivíduo ao convívio social fora das grades, onde as normas de conduta coletiva são invariavelmente diferentes. Ademais, se considerarmos que o preso deve adotar uma série de comportamentos a fim de manter sua integridade física e psicológica nas penitenciárias, torna-se patente que o sistema penal brasileiro não cumpre seu papel principal: o da ressocialização do detento \citep{silva2011visao}. O política penitenciária nacional, portanto, está em estado de crise permanente.

Outro trabalho pioneiro é o de \citet[]{ramalho1979mundo}. O autor desenvolveu longa pesquisa de campo na maior prisão de São Paulo, a extinta Casa de Detenção do Carandiru, e trouxe uma inovação para os estudos prisionais: uma análise espacial da penitenciária. \citeauthor[]{ramalho1979mundo} destaca a forte dicotomia entre o ``trabalho'' e o ``crime'' no Carandiru. Detentos que trabalhavam no presídio eram considerados aptos para voltarem à vida social em um futuro próximo, e geralmente viviam em pavilhões com melhor estrutura. Por outro lado, os criminosos reincidentes eram vistos como ``irrecuperáveis'' pela administração penitenciária, e viviam nos espaços mais insalubres da cadeia. A divisão social do presídio se refletia também na utilização do espaço carcerário.

\citet[]{coelho1987oficina} completa a tríade dos textos clássicos sobre as gangues prisionais brasileiras. Escrito quando o governo do Rio de Janeiro tentou estabelecer canais de comunicação com os detentos, o livro mostra como as ``comissões dos presos'' foram logo dominadas por líderes locais e utilizadas como um instrumento de opressão contra os demais presos. Além desse problema, o forte aumento nas taxas de homicídio no Rio (sobretudo devido às disputas territoriais de traficantes de drogas) aumentou a pressão sobre os governadores para que eles adotassem uma abordagem mais dura contra os presidiários, o que realmente acontece a partir da década de 1990.

\citet[]{salla2007montoro} descreve a evolução das políticas penitenciárias no estado de São Paulo de 1982 a 2005. \citeauthor[]{salla2007montoro} nota que, embora o Brasil estivesse atravessando um processo de democratização durante o período, as tentativas de modernizar o sistema carcerário não foram bem-sucedidas. Enquanto Franco Montoro tentara adotar uma abordagem ``mais humana'' no trato com os detentos nos anos 1980, as reformas foram desmontadas pelos governos Quércia e Fleury na década seguinte. O ápice dessa política repressiva, conforme assinalado pelo autor, foi o conhecido massacre do Carandiru de 1992, na qual 103 presos foram mortos pelas forças policiais e que levou à criação do PCC como resposta a este ato brutal de repressão governamental. \citeauthor[]{salla2007montoro} nota que ainda não há um pensamento de longo prazo para as políticas públicas carcerárias, e ainda nos dias de hoje as mudanças penitenciárias são costumeiramente \textit{ad hoc}, incrementais e tímidas \citep[383]{dias2011pulverizaccao}.

\citet[]{alvarez2013comissoes} trazem outro estudo sobre as recentes políticas prisionais do estado de São Paulo. Os autores sugerem a hipótese de que o surgimento das gangues carcerárias em São Paulo são fruto da ausência de canais oficiais de comunicação entre os presos e as autoridades públicas. Apenas com o uso da força os detentos conseguiram trazer o governo para a mesa de negociação. Reafirmando o argumento de \citet[]{salla2007montoro}, os autores afirmam que o surgimento do PCC está atrelado à violência institucional contra os detentos nos anos 1990, na qual a facção surge como um possível mediador entre os interesses dos encarcerados e os órgãos administrativos.

\citet[]{biondi2010junto} apresenta uma densa etnografia de um presídio dominado pelo PCC. Ela apresenta uma breve cronologia da facção e descreve a ``ética criminosa'' do grupo com detalhes. A autora aponta que os princípios dominantes na organização do PCC são a promoção da igualdade entre seus membros e a garantia dos direitos dos presos perante as autoridades, ainda que para isso seja necessário o uso da violência. Entretanto, como mencionado por \citet[376]{dias2011pulverizaccao}, ao aceitar o discurso oficial do PCC sem maiores críticas, \citeauthor[]{biondi2010junto} não dá a devida atenção às constantes agressões perpetradas pelo grupo, como a violência contra usuários de drogas, homossexuais e estupradores, o que dá margem a uma série de questionamentos quanto ao discurso de ``integração e igualdade'' alimentado pela facção.

O texto de \citet{dias2011pulverizaccao}, resultado de sua pesquisa de doutorado, destaca-se como o mais completo trabalho sobre o PCC até hoje. O texto mapeia as origens e expansão do grupo e, fazendo uso de trabalho de campo em três presídios paulistas, fornece subsídios empíricos para embasar os argumentos apresentados em sua monografia. Entretanto, há ainda uma enorme carência de trabalhos analíticos sobre o PCC: apesar da autora ter esmiuçado a história da facção, descrevendo em detalhes as distintas fases do grupo, ela não desenvolve um arcabouço analítico capaz de extrapolar os achados e torná-los úteis para analisar outras organizações criminosas que se encontram em condições similares.  

A literatura sobre o PCC deixa grandes lacunas na descrição de seu objeto de pesquisa. 
Como mencionado anteriormente, embora haja uma literatura relativamente vasta sobre os problemas da jurídicos da administração prisional e da facção, os textos vêm tratando apenas marginalmente das questões ligadas à governança das gangues prisionais. Com efeito, eles se silenciam a respeito de como os grupos empregam a violência para fazer cumprir seus regulamentos internos. Os trabalhos recentes também não resolvem o paradoxo que, embora os detentos estejam inescapavelmente impedidos de utilizar qualquer estratégia perfeitamente racional por causa da marcada incerteza de seu ambiente, ainda assim as gangues prisionais são capazes de utilizar, de modo bastante eficiente, diversos tipos de violência seletiva.

\section{Metodologia}

\subsection{Referências Teóricas}

O presente projeto de pesquisa traz uma abordagem teórica inédita para as pesquisas sobre as gangues prisionais e, por conseguinte, para os estudos sobre o PCC. Três correntes fornecerão subsídios para o trabalho em questão: a teoria de sinalização, escolha pública e sociologia analítica. Discutiremos abaixo como cada uma delas irá colaborar para cumprirmos os objetivos de pesquisa acima expostos e colocar as práticas do PCC sob uma perspectiva diferente.

A \textit{teoria de sinalização} é o primeiro marco teórico a ser utilizado na futura tese. Originalmente desenvolvida na economia, a teoria de sinalização foi trazida para as demais áreas sociais ainda nos anos 1970 e desde então tem se mostrado como um dos ramos mais profícuos da ciência em geral, sendo hoje empregada em áreas tão diversas quanto a antropologia e a zoologia \citep{connelly2011signaling, cook2007cooperation}. Em termos gerais, a teoria busca entender como dois agentes com diferentes níveis de informação se comunicam, ou, em outras palavras, como um agente pode transmitir um sinal a respeito de uma característica não-observada que não somente seja claramente compreensível, mas que chegue ao receptor correto e não possa ser falsificado por impostores \citep{gambetta2009signaling}. Um exemplo clássico é o das habilidades intrínsecas de um potencial candidato a um posto de trabalho, as quais podem ser sinalizadas pela qualidade de sua educação formal, uma vez que o custo para obter uma boa educação é sabidamente alto e não pode ser copiado com facilidade \citep{spence1973job}. Este tipo de problema é crucial para entendermos o crescimento dos grupos prisionais e sua capacidade de monopolizar a violência carcerária. Criminosos em geral, e detentos em particular, enfrentam sérios problemas de comunicação \citep{freire2014entering, gambetta2009codes}. Como o custo de enviar uma informação errada é quase sempre a morte, não pode haver erro na sinalização da parte de ambos os agentes, a gangue e os potenciais membros. No caso do PCC, pode-se imaginar que esse sinal só fora enviado pelo grupo de inequívoco após as rebeliões de 2001 e 2006\footnote{Neste exemplo, o grupo é tomado como agente. É também possível, em situações como a de recrutamento, que os detentos sejam os agentes e o PCC os receptores. Para uma análise desse processo usando modelagem matemática, ver \citet{freire2014entering}.}. Com o sucesso das revoltas, o grupo mandara um sinal confiável de que é eficiente e destemido, \textit{já que só uma organização competente conseguiria realizar um ato como aquele}. Desse modo, a teoria de sinalização
permite-nos avaliar um tópico ainda pouco analisado nos estudos de crime em geral e do PCC em particular: \textit{como os criminosos se comunicam entre si e constroem laços de confiança mútua}. A comunicação do grupo, com efeito, é a chave para entendermos como os detentos se articulam e criam um tipo de ordem em condições notavelmente adversas. A teoria de sinalização pode, dessa maneira, explicar os motivos pelos quais o PCC conseguiu, de modo pioneiro, congregar os presos paulistas e criar laços com grupos criminosos fora dos presídios, fatores determinantes para seu sucesso.

O trabalho também trará a escolha pública (\textit{public choice theory}) como referência. Essa escola de pensamento integra-se facilmente à teoria de sinalização. A escolha pública, cujo intuito explícito é ``lançar mão de ferramentas econômicas para estudar fenômenos da ciência política''\footnote{A definição foi dada por um dos criadores da desta teoria, Gordon Tollock, e consta no \textit{The New Palgrave Dictionary of Economics}. O artigo está disponível no seguinte endereço eletrônico: \href{http://www.dictionaryofeconomics.com/article?id=pde2008_P000240}{http://www.dictionaryofeconomics.com/article?id=pde2008\_P000240}. Acesso: 10 de janeiro de 2015.}, tem como um de seus pilares a análise de decisão coletiva e os problemas de coordenação que estas decisões implicam \citep{abrams1980foundations, arrow1951social}. Dos vários escritos dessa corrente, \citet{olson1965logic} é provavelmente o que mais facilmente se presta ao estudo das gangues prisionais, e já vem sido mobilizado por alguns autores do campo, notadamente por \citet{skarbek2011governance, skarbek2014social}. Rompendo com a antiga noção de que interesses comuns seriam uma condição suficiente para levar a ação coletiva, \citet[2]{olson1965logic} afirma: 

\begin{quote}
[\dots] a menos que o número de indivíduos seja pequeno, ao menos que haja coerção ou outro instrumento específico que faça os indivíduos agirem em seu interesse comum, \textit{indivíduos racionais e auto-interessados não irão agir para obter seus interesses comuns ou de grupo}\footnote{Essa hipótese é geralmente chamada de \textit{zero contribution thesis} \citep{ostrom2000crowding}.}. (grifos do autor)
\end{quote}

A resposta, diz o autor, estaria na provisão de bens públicos a serem desfrutados pelos membros do grupo. Entretanto, nenhum dos envolvidos têm incentivos para proverem eles mesmos tais bens coletivos. Vários autores buscaram saídas para resolver este paradoxo, sendo que o próprio \citet{olson1965logic} sugeriu uma saída: o fornecimento de bens privados, que só alguns podem ter acesso. Com isso, os indivíduos teriam algo especial a ganhar, o que daria motivação suficiente para iniciarem o processo de ação coletiva.
\citet{skarbek2011governance} argumenta que, no caso das gangues prisionais, esses bens privados seriam basicamente segurança no presídio e uma parcela da taxação que as gangues impõem a seus associados. Uma vez que os criminosos sabem que cedo ou tarde serão apanhados e detidos, eles preferem antecipar esse problema -- o encarceramento --
e colaborar com o financiamento da gangue. Nesse sentido, tanto a facção como os demais criminosos teriam interesses congruentes, o que levaria à ação coletiva. 

É válido notar que o argumento exposto aqui, ainda que sofisticado, não é o único possível para explicar a coordenação de agentes. Essa linha de raciocínio simplifica o processo de ação coletiva e trata todos os agentes como estritamente racionais e perfeitamente informados, embora essas condições nem sempre sejam vistas na realidade \citep{baldassarri2009collective}. \citet{ostrom2000collective} dá um passo além das explicações puramente racionalistas e incorpora \textit{normas sociais} aos debates de ação coletiva. A autora afirma que a interdependência entre os atores faz com que as normas de ação ganhem importância mesmo em um cenário onde \textit{a priori} todos são racionais e egoístas, e defende, com dados experimentais, a tese de que ``[\dots] as normas podem levar os indivíduos a comportarem-se diferentemente em uma mesma situação objetiva, dependendo do quão forte eles valorizam o cumprimento da norma'' \citep[144]{ostrom2000collective}. Esta abordagem ``evolutiva'' adotada pela autora mostra que o comportamento altruísta também é possível, e essa diferenciação funcional para o bem do grupo é um resultado das \textit{interações contínuas} entre os indivíduos, de modo similar ao que ocorre nos presídios\footnote{\citet{ostrom2000collective} identifica ao menos três tipos ideais de atores que podem agir nas ações coletivas, a saber: \textit{rational egoists}, os maximizadores de utilidade que aparecem em \citet{olson1965logic}, \textit{conditional cooperators}, que iniciam ações cooperativas e continuam a cooperar se há um número mínimo de indivíduos a seu lado, e os \textit{and willing punishers/rewarders}, que voluntariamente pune os caroneiros e recompensa quem contribuiu mais do que necessário.} \citep[144]{ostrom2000collective}. Certos mecanismos normativos utilizados pelas gangues já foram identificados na literatura, e atenção especial foi dada a dois deles: as constituições -- que disseminam informação para todos os detentos e explicitam os custos e benefícios de se filiar a uma facção \citep{leeson2010criminal} -- e a ``cultura prisional'', certos hábitos de conduta transmitidos oralmente entre os detentos \citep{irwin1962thieves}. 

Contudo, o papel de tais mecanismos na ação coletiva no PCC é ainda desconhecido. Não apenas as pesquisas acadêmicas ignoraram esta dimensão da vida prisional até o momento, como o instrumental teórico utilizado nos trabalhos recentes sobre o assunto não são adequados para avaliar esses argumentos. A ideia de atores heterogêneos e condutas normativas, por exemplo, não pode ser estimada pela teoria dos jogos, o que abre espaço para modelagem baseada em agentes, como será discutido na seção seguinte. Além disso, uma vez que os textos sobre o PCC focam-se quase que exclusivamente nos aspectos jurídicos ou antropológicos da existência da facção \citep[365]{dias2011pulverizaccao}, não há um debate aprofundado sobre aspectos organizacionais do grupo.

Por fim, a tese fará uso da sociologia analítica. Área em rápida expansão nas ciências sociais, a sociologia analítica é uma estratégia de pesquisa que busca explicar os fenômenos coletivos \textit{por meio de mecanismos causais individuais} \citep{hedstrom1998social}. Mecanismos causais consistem em uma série de processos que demonstram a relação generativa entre o \textit{explanans} e o \textit{explanandum} \citep[5]{little1991varieties}, e têm como função teórica distinguir relações causais verdadeiras de correlações espúrias em dados não-experimentais \citep[54]{hedstrom2010causal}. De preferência, buscam-se mecanismos que possam ser testados contrafactualmente e que, por meio de simulações computacionais, sejam avaliados em diferentes cenários plausíveis \citep{macy2002factors}. Esta condição liga naturalmente a sociologia analítica à modelagem baseada em agentes, técnica que discutiremos na próxima seção deste projeto.

A sociologia analítica também visa um compromisso entre as abordagens estruturalistas e o individualismo metodológico, argumentando não apenas que os fatos sociais só podem ser entendidos com referências a seus agentes constitutivos, mas também que tais agentes são influenciados pelas relações de interdependência que mantêm com os demais atores \citep[4]{demeulenaere2011analytical}. Efetivamente, a sociologia analítica forma o que se pode chamar de ``individualismo estruturalista'', na qual os agentes são embuídos de preferências socialmente construídas e, embora focada nas explicações em nível individual, fatores como hábitos culturais e normas sociais são trazidas para o centro da pesquisa \citep{hedstrom2009analytical}. Com isso, é possível articular os fatores micro e macro em uma ``teoria de alcance médio'' (\textit{middle range theory} nos termos de \citet{merton1973sociology}), conciliando os extremos da descrição densa -- sabidamente pouco generalizável -- e as teorias puramente abstratas, como a de escolha racional \citep{little2012explanatory}.

As três teorias aqui apresentadas dialogam facilmente entre si. Enquanto a sociologia analítica fornece o instrumental teórico de base e as técnicas de pesquisa adequadas para o projeto (dados qualitativos articulados com modelagem computacional e estatística), a teoria de ação coletiva nos ajuda a entender os condicionantes em nível agregado e a teoria de sinalização dispõe de excelentes mecanismos causais para problemas de informação e outros componentes relevantes para os indivíduos. Fecha-se assim um consistente modelo teórico para entendermos a atuação do Primeiro Comando em São Paulo, o qual também pode ser mobilizado para entendermos outros grupos em situação semelhante dado que as estruturas e mecanismos sejam semelhantes. Com isso, pode-se expandir o diálogo entre a experiência brasileira e a literatura sobre violência no exterior e, também, contribuir para o refinamento das teorias supracitadas por meio de um teste de seus pressupostos com o caso do PCC. 

\subsection{Técnicas de Pesquisa}

A tese empregará métodos mistos a fim de resolver as questões propostas acima. A idéia é utilizar aquilo que \citet[165]{laitin2003perestroikan} famosamente descreveu como ``metodologia tripartite'' para as ciências sociais, isto é, integrar em um mesmo estudo narrativas qualitativas, análise estatística e modelagem formal. Se os estudos de caso são, tradicionalmente, a ferramenta teórica mais comum nos estudos sobre crime organizado no Brasil \citep[366]{dias2011pulverizaccao}, as outras duas técnicas ainda têm uso incipiente no País e representam assim uma inovação na área. 

A parte qualitativa da futura pesquisa será conduzida em duas etapas, a primeira delas consistindo em trabalho de campo. Pretende-se visitar ao menos três prisões, sendo duas controladas pelo PCC e uma dominada por uma gangue rival ou que seja ``independente''.  Três prisões é o número mínimo necessário para satisfazer os requisitos das análises dos sistemas ``mais similares'' e ``mais diferenciados'', tais quais sugeridos por \cite{przeworski1970logic}, e o fato de ter uma prisão que não é controlada pela facção nos permite ter um ponto de vista diferente para os dados fornecidos pelos presos nas cadeias do Comando. Dessa forma, podemos verificar as informações obtidas por diferentes grupos, e assim atestar a plausibilidade dos relatos que nos serão apresentados. 

As prisões serão escolhidas de acordo com dois critérios, um metodológico e outro de natureza prática. O critério prático que guiará a seleção dos casos é o acesso às instalações penitenciárias. A primeira aproximação se dará por meio de contato com pesquisadores e centros de estudos que já realizaram trabalho de campo em presídios. O primeiro contato será feito com Sasha Darke, pesquisador inglês com vasto trabalho de campo em presídios brasileiros e que trabalha atualmente em projeto conjunto sobre encarceramento com meu orientador principal, David Skarbek. Feito este contato, segue-se por via de \textit{snowballing} até os membros do poder público que podem agilizar os procedimentos. Embora acadêmicos tenham notado que a administração penitenciária e os detentos têm geralmente colaborado com os pesquisadores \citep[35]{dias2011pulverizaccao}, uma miríade de autorizações, formais e informais, são necessárias para adentrar nesses espaços. Esta questão terá um impacto sobre a seleção dos casos.

A segunda parte dos estudos qualitativos consistirá em entrevistas em profundidade. Após estarem definidas as questões específicas, formuladas a partir dos dados colhidos na primeira fase, entrevistas focadas em certos temas permitirão uma abordagem mais aprofundada dos problemas a serem investigados. A ideia é conduzir entrevistas não apenas com membros do PCC mas também com presos que não fazem parte da facção, associados de outros grupos criminosos e agentes penitenciários. A técnica a ser utilizada será o \textit{snowball sampling} \citep{goodman1961snowball}, na qual um primeiro grupo de entrevistados indica, dentre sua rede de contatos, outros participantes que podem ser relevantes para responder a determinadas questões. Apesar de não ser uma técnica amostral probabilística, o \textit{snowball sampling} presta-se bem ao tipo de trabalho que pretendemos desenvolver com grupos criminosos, uma vez que a credibilidade dos primeiros contatos permite que novos informantes sejam incorporados à pesquisa.

Em seguida, a pesquisa lançará mão de modelagem baseada em agentes (\textit{agent-based modeling}) para generalizar os achados da primeira fase. Utilizada há décadas nas ciências naturais \citep{zeigler2000theory}, a modelagem formal vem sido altamente recomendada em vários ramos das ciências sociais, especialmente quando a interação entre os agentes é complexa e as dinâmicas individuais podem se transformar, a nível macro, em fenômenos distintos dos originalmente planejados \citep{hedstrom2005dissecting, mccarty2007political, ordeshook1986game}. A modelagem baseada em agentes, por ser  simulações computacionais, permite uma descrição muito mais rica dos fenômenos do que a teoria formal tradicional. Por um lado, a modelagem computacional relaxa as premissas controversas da \textit{rational choice}, tais como o cálculo objetivo de utilidade e a estimação precisa de probabilidades \citep{bonabeau2002agent}, e permite a inclusão de normas sociais, heurísticas e mesmo dinâmicas simbólicas coletivas nas análises\footnote{As normas podem ser incluídas com simples programação condicional, utilizando comandos do tipo \texttt{if-then-else}.} \citep{gilbert2000build}. Assim, os modelos podem ser muito mais realistas do que aqueles utilizados na teoria dos jogos\footnote{Como sugere \citet[124]{abbott2001time}, ``a teoria dos jogos não nos levará longe, pois ela é alheia, exceto nos termos mais gerais, a uma preocupação séria com a estrutura e complexos efeitos temporais. Mas a simulação pode nos ajudar a entender os limites e as possibilidades de certos tipos de áreas inter-relacionais, e isso seria conhecimento profundamente sociológico''.} \citep{axelrod1997complexity, epstein2006generative}, mas ainda manter a clareza explicativa necessária para abrirmos a ``caixa preta da causalidade'' e elucidarmos os mecanismos causais que ligam as ações individuais aos resultados agregados, como prega a sociologia analítica \citep[13]{demeulenaere2011analytical}. Com efeito, a modelagem permite que fatos sociais sejam explicados não apenas por correlação a outros fatos sociais, mas trazendo argumentos claros e mecanismos causais factíveis para entendermos o mundo social \citep{de2005computational}. 

Há ainda outra vantagem no uso de experimentos computacionais nas ciências humanas. Por definição, a realidade empírica é \textit{uma e apenas uma} materialização da vasta gama de possíveis respostas a determinados fenômenos; já a modelagem computacional permite ao pesquisador testar por simulação todos os contrafactuais que desejar, não apenas avaliando quais os parâmetros seriam realmente relevantes em diversas situações hipotéticas, mas também cumprindo todos os requisitos teóricos do modelo mais exigente de inferência causal, o modelo causal de Rubin \citep{holland1986statistics, imbens2009causal, morgan2007counterfactuals, rubin2005causal}. 

Com efeito, modelagem computacional é um rigoroso e importante instrumento para avaliar argumentos dedutivos e podem levar a resultados interessantes e generalizáveis. Apesar de seu grande potencial, o emprego de modelagem baseada em agentes ainda é muito tímido nos estudos criminais. Excetuando-se o trabalho pioneiro de \citet{dixit2011game}, no qual o autor traduz em modelagem formal alguns dos argumentos desenvolvidos por \cite{gambetta2009codes}, muitos dos dados coletados sobre as gangues prisionais ainda estão por ser testados com novas metodologias\footnote{Curiosamente, Elinor Ostrom e Thomas Schelling, ambos laureados com o Prêmio Nobel de Economia e autores fundamentais, respectivamente, das teorias de ação coletiva e de sinalizacão mencionadas neste projeto, são notórios defensores da modelagem computacional e fizeram largo uso dessa metodologia em suas carreiras acadêmicas. Ver, entre outros, \citet{janssen2006empirically} e \citet[xi]{schelling2006strategies}.}. No que tange ao PCC, não há \textit{um único estudo} que usa modelagem para entender as operações do grupo, embora tais métodos são claramente adequados para problemas com informações imperfeita\footnote{Os estudos de terrorismo, por exemplo, são ricos em estudos formais. Ver, entre outros, \citet{de2005quality} e \citet{fricker2006game}.}. Os motivos são vários. Dado que diversas características das organizações criminosas não são diretamente observáveis, a modelagem coloca tais argumentos teóricos e sugestões \textit{a priori} em uma estrutura rigorosa e permite o teste de argumentos contrafactuais em vários cenários possíveis \citep{baldassarri2009collective}. 

Adicionalmente, pretendo incluir análises estatísticas a fim de testar a validade externa dos achados das primeiras duas etapas. Apesar dos métodos estatísticos provavelmente ocuparem um lugar menor na pesquisa, dado que sua obtenção é custosa e incerta, eles são importantes para estimar o impacto relativo das variáveis independentes e identificar possíveis \textit{outliers}. O primeiro modelo será uma estimação binomial negativo testando o impacto nos índices de violência letal nos presídios de uma variável \textit{dummy} indicando presença do PCC, somado a uma série de covariantes sócio-demográficas tradicionalmente presentes nos estudos do tema para controlar efeitos espúrios \citep{delisi2004gang}. Afim de se reduzir a dependência da forma funcional do modelo estimado nos resultados, pretende-se pré-processar os dados obtidos usando técnicas não-paramétricas de \textit{matching} e assim balancear os casos de tratamento e controle \citep{ho2007matching}. O uso de \textit{matching} permite aos pesquisadores obter inferências causais não-enviesadas a partir de dados observacionais \citep{angrist2008mostly}.

Planejo também realizar uma pesquisa de survey com um número razoável de detentos (ao menos com $n \geq 30$) e testar hipóteses mencionadas acima com dados desses questionários. Seguindo o modelo de survey prisionais indicado por \citet{winterdyk2010managing}, a pesquisa testaria quais as razões pelas quais os detentos se uniram ao PCC. Os autores sugerem as seguintes categorias para os questionários: acesso a contrabando; benefícios econômicos; sentimento de pertencimento; \textit{status}; intimação/medo de outros presos \citep[734]{winterdyk2010managing}. Estas alternativas captariam corretamente o espectro das escolhas mais comuns e, também, captariam as variações no tempo, uma vez que se espera que os motivos, na média, tenham se alterado nas diferentes fases do PCC.

\section{Justificativa para Realização de Pesquisa no Exterior}

A maior contribuição do presente projeto de pesquisa está no uso de referências e técnicas ainda incomuns nos estudos de violência no Brasil. Nesse sentido, conduzir o trabalho no exterior é parte central da estratégia para que o doutorado seja bem-sucedido. Uma vez que as teorias utilizadas não são parte da grade curricular de qualquer departamento de ciência política nacional, a necessidade de treinamento estrangeiro é fundamental.

Tome-se, por exemplo, os casos da sociologia analítica e da modelagem baseada em agentes. A sociologia analítica é um ramo científico que nasceu na Inglaterra -- tendo o King's College London e a Universidade de Oxford colaborado decisivamente nesta direção -- porém ainda não existe \textit{nenhum} curso no Brasil sobre o assunto em nível superior. Uma pesquisa no Google Scholar em português usando ``sociologia analítica'' e ``mecanismos causais'' como palavras-chave retorna \textit{apenas um} artigo mencionando o tema, e este ainda assim o faz em referência a outro tópico (estudos de caso na ciência política)\footnote{Ver: A pesquisa pode ser vista no seguinte endereço: \href{http://scholar.google.com.br/scholar?hl=en&q=\%22sociologia+anal\%C3\%ADtica\%22+\%22mecanismos+causais\%22&btnG=&as_sdt=1\%2C5&as_sdtp=}{http://scholar.google.com.br/scholar?hl=en&q=\%22sociologia+anal\%\\C3\%ADtica\%22+\%22mecanismos+causais\%22&btnG=&as\_sdt=1\%2C5&as\_sdtp=}. Acesso: 10 de janeiro de 2015.}.

As dificuldades são as mesmas com a modelagem baseada em agentes. Não apenas os programas de doutorado nacionais são bastante focados em metodologia qualitativa, como a modelagem computacional ainda não aparece sequer como curso extra para os interessados da ciência política. Por exemplo, o curso de verão em métodos de pesquisa promovido pela Universidade de São Paulo em parceria com a International Political Science Association -- um dos mais diversos e internacionalizados do País -- nunca teve uma disciplina de modelagem computacional\footnote{Ver: \href{http://summerschool.fflch.usp.br/course-offerings/courses}{http://summerschool.fflch.usp.br/course-offerings/courses}. Acesso: 10 de janeiro de 2015.}. Contudo, como discutimos acima, essa metodologia é extremamente promissora para a ciência política \citep{janssen2006empirically} e vem sendo empregada em diversos estudos no exterior\footnote{No Brasil, destaca-se o pioneiro trabalho de Bruno Reis (UFMG).}, de segregação espacial \citep{clark2008understanding} a trabalhos sobre genocídio \citep{bhavnani2006ethnic, srbljinovic2003agent}, de difusão cultural \citep{axelrod1997dissemination} a guerras civis \citep{epstein2002modeling}. A presença de uma sociedade científica na Europa totalmente dedicada ao ensino e difusão de modelos computacionais para ciências humanas, a European Social Simulation Association\footnote{Mais informações em: \href{http://www.essa.eu.org/}{http://www.essa.eu.org/}. Acesso: 11 de janeiro de 2015.}, também colabora nesse sentido.

Um terceiro motivo para concluir meu doutorado no exterior surge da importância de internacionalizar o debate a respeito dos problemas brasileiros. Esta razão é ao menos tão importante quanto as duas primeiras. O debate sobre o PCC e a situação penitenciária do Brasil é feito, com raríssimas exceções\footnote{Ver, por exemplo, \citet[]{darke2013inmate, king2014power}.}, todo em português e apenas nas instituições nacionais. A internacionalização deste campo de pesquisa traz ao menos dois benefícios notórios para o campo: por um lado, aumenta significativamente o impacto dos textos de autores brasileiros, casos estes sejam publicados em revistas estrangeiras de alta visibilidade; por outro, ao realizar um trabalho como o aqui apresentado no exterior, acadêmicos de renome internacional que já trabalham com gangues carcerárias passam a colaborar para as pesquisas a respeito do caso brasileiro, enriquecendo assim a produção teórica sobre um assunto de grande importância para as políticas públicas nacionais. Podemos, assim, dar os primeiros passos para a formação uma comunidade internacional de pesquisadores sobre gangues prisionais brasileiras. 

É importante, contudo, assinalar o papel crucial que o King's College London terá nesse processo. A faculdade é produtora de conhecimentos de ponta em todas as áreas mencionadas neste trabalho. Segue um breve comentário sobre a relevância da universidade para minha pesquisa.

\section{Justificativa para Indicação do King's College London}

Tenho a mais firme convicção de que o King's College London é a melhor instituição para realizar a pesquisa descrita neste projeto. Primeiramente, o King's College tem uma inquestionável reputação acadêmica, e está hoje classificada como a 16\textsuperscript{a} melhor universidade do mundo pelo \textit{QS World University Rankings}\footnote{Dados disponíveis em \href{http://www.topuniversities.com/university-rankings/world-university-rankings/2014}{http://www.topuniversities.com/university-rankings/world-university-rankings/2014}. Acesso: 16 de dezembro de 2014.}. O departamento no qual realizarei meu trabalho é igualmente bem avaliado:  de acordo com a última avaliação de ciências sociais do \textit{Times Higher Education}, o King's College London obteve a 25\textsuperscript{a} posição no ranking global\footnote{A lista completa pode ser vista em: \href{http://www.timeshighereducation.co.uk/world-university-rankings/2012-13/subject-ranking/subject/social-sciences}{http://www.timeshighereducation.co.uk/world-university-rankings/2012-13/subject-ranking/subject/social-sciences}. Acesso: 16 de dezembro de 2014.}, o que claramente coloca seu nome dentro dos centros de excelência na área.

Ademais, o Departamento de Economia Política é uma instituição interdisciplinar única no Reino Unido, na qual cientistas políticos e economistas trabalham em conjunto em uma série de problemas de pesquisa. Como o departamento exige que todos os alunos escolham dois orientadores para sua tese (sendo um deles o principal), o caráter multidisciplinar do curso está presente não apenas nas aulas, mas é também colocado em prática na orientação dos estudantes durante seu mestrado ou doutorado.

No que tange ao meu tópico de interesse, gangues prisionais, o Departamento de Economia Política possui dois grupos temáticos nos quais meu projeto estará inserido: ``Racionalidade, Escolhas e Incerteza'' e ``Regulamentação, Governança e Ordem''\footnote{Mais informações sobre os grupos de pesquisa estão disponíveis em: \href{http://www.kcl.ac.uk/sspp/departments/politicaleconomy/research/Research-groups/Research-Groups.aspx}{http://www.kcl.ac.uk/sspp/departments/political\\economy/research/Research-groups/Research-Groups.aspx}. Acesso: 02 de janeiro de 2015.}. Ambos os grupos já foram previamente contactados e mostraram-se interessados em meu projeto de pesquisa\footnote{Comunicação eletrônica com o autor. As mensagens estão disponíveis se necessário.}. O King's College também abriga um dos centros de estudos brasileiros mais ativos da Europa, o \textit{King's Brazil Institute}. O centro frequentemente trabalha em parceria com o Departamento de Economia Política, e é atualmente dirigido pelo professor Anthony Pereira, membro do comitê executivo da \textit{Brazilian Studies Association}. 

Faz-se necessário destacar o papel de meu orientador principal, professor David Skarbek, na realização do projeto aqui exposto. Skarbek é um conhecido pesquisador das gangues prisionais e vem escrevendo textos de inquestionável valor acadêmico sobre o tema, muitos valendo-se de estudos de caso e, também, de modelagem formal. Vários artigos recentes do professor foram publicados em conhecidos periódicos da área, como a \textit{Public Choice}, \textit{Rationality \& Society}, \textit{American Journal of Economics \& Sociology} e, inclusive, na \textit{American Political Science Review}, a revista de maior prestígio da ciência política mundial\footnote{Ver: \href{http://journals.cambridge.org/action/displayMoreInfo?jid=PSR&type=if}{http://journals.cambridge.org/displayMoreInfo?jid=PSR\&type=if}. Acesso: 18 de dezembro de 2014.}. Em três anos, esse último texto já recebeu mais de 50 citações, o que mostra a relevância desse trabalho nos estudos sobre gangues prisionais. Em julho passado o professor publicou seu \textit{The Social Order of the Underworld}, editado pela Oxford University Press, no qual descreve como as gangues americanas exercem sua influência dentro e fora do sistema prisional e como elas estabelecem suas regras em um dos mais severos sistemas carcerários do mundo. O livro recebeu resenhas elogiosas de diversos acadêmicos de renome, como David Laitin, professor titular de Ciência Política na Universidade de Stanford, Philip Keefer, economista-chefe do Banco Mundial e de Thomas Schelling, pioneiro da modelagem baseada em agentes e Prêmio Nobel de Economia em 2005\footnote{Ver \href{http://amzn.to/QoVafa}{http://amzn.to/QoVafa}. Acesso: 18 de dezembro de 2014.}. Além disso, a obra ganhou o \textit{Outstanding Publication Award} em 2014, a maior distinção da \textit{International Association for the Study of Organized Crime}\footnote{Ver: \href{http://www.davidskarbek.com/book.html}{http://www.davidskarbek.com/book.html}. Acesso: 02 de janeiro de 2015.}. Por fim, o livro suscitou um vigoroso debate fora da academia, e veículos conhecidos como a revista \textit{The Atlantic}\footnote{Ver: \href{http://www.theatlantic.com/features/archive/2014/09/how-gangs-took-over-prisons/379330/}{http://www.theatlantic.com/features/archive/2014/09/how-gangs-took-over-prisons/379330/}. Acesso:  02 de janeiro de 2015.}, o jornal \textit{The Independent}\footnote{Ver: \href{http://www.independent.co.uk/arts-entertainment/books/reviews/the-social-order-of-the-underworld-by-david-skarbek-book-review-a-troubling-study-of-death-and-survival-in-us-prisons-9823750.html#}{http://www.independent.co.uk/arts-entertainment/books/reviews/the-social-order-of-the-underworld-by-david-skarbek-book-review-a-troubling-study-of-death-and-survival-in-us-prisons-9823750.html#}. Acesso:  02 de janeiro de 2015.} e a \textit{The Economist}\footnote{Ver: \href{http://www.economist.com/news/books-and-arts/21614090-prison-gangs-are-rational-solution-growing-problem-protection-rackets}{http://www.economist.com/news/books-and-arts/21614090-prison-gangs-are-rational-solution-growing-problem-protection-rackets}. Acesso:  02 de janeiro de 2015.}, trouxeram editoriais comentando os resultados da pesquisa do professor. 

Professora Anja Shortland, co-orientadora do meu futuro projeto, também possui grande expertise em modelagem baseada em agentes e um excelente currículo acadêmico. Consultora do Banco Mundial por vários anos, Shortland hoje dedica-se a pesquisas sobre organizações rebeldes e economia do crime utilizando-se de métodos quantitativos e simulações computacionais como as que pretendo empregar. Sua publicação mais recente, por exemplo, recebeu o Prêmio Nils Petter Gleditsch de melhor publicação do ano no tradicional \textit{Journal of Peace Research}, importante revista na área de conflitos internos e internacionais. A revista destacou ``[...] o rigor teórico, a sofisticação metodológica e a relevância substantiva'' do trabalho\footnote{Disponível em: \href{http://jpr.sagepub.com/site/ArticleoftheYearAwards/JPR_AoY_2013.pdf}{http://jpr.sagepub.com/site/ArticleoftheYearAwards/JPR\_AoY\_2013.pdf}. Acesso: 19 de agosto de 2014.}. 

Em resumo, o King's College London e seu Departamento de Economia Política são uma excelente escolha para conduzir minha pesquisa. Instituição de inegável prestígio acadêmico e com claro interesse em temas brasileiros, o King's College parece-me a escolha ideal para meu doutorado, e a supervisão do professor David Skarbek encaixa-se perfeitamente na minha proposta acadêmica. Após ter recebido a oferta incondicional do departamento, estando assim satisfeitos todos os requisitos para admissão, incluindo os de língua inglesa, percebe-se que instituição também possui interesse em meu projeto. Espero, desse modo, atender às expectativas do departamento, completar meu doutorado com sucesso e posteriormente contribuir com o desenvolvimento acadêmico brasileiro quando retornar ao País.

\section{Cronograma}

O cronograma sugerido abaixo segue todos os critérios recomendados pela CAPES. Ressalte-se que o trabalho de campo (que aqui consiste nas entrevistas e questionários) está agrupado em um único semestre e será realizado após a conclusão do Master of Philosophy (primeiro ano) como sugere o edital. Ele também será concluído antes do último ano do curso e não coincide com o período inicial da bolsa.

\vspace{1cm}
{\scriptsize \begin{tabular}{|c|c|c|c|c|c|c|c|}
\hline \textbf{Período/} & 3\textsuperscript{\underline{o}} Trim. & 4\textsuperscript{\underline{o}} Trim. & 1\textsuperscript{\underline{o}} Trim.  & 2\textsuperscript{\underline{o}} Trim. & 3\textsuperscript{\underline{o}} Trim. & 4\textsuperscript{\underline{o}} Trim.\\ 
\textbf{Atividade} & 2015 & 2015 & 2016 & 2016 & 2016 & 2016\\ 
\hline Aulas & X & X & X& X& &\\
\hline Pesquisa/Monitoria no King's & X & X & X& X& & \\
\hline Análise Bibliográfica & X & X & & & & \\
\hline Modelagem Formal &  &  &  &  &  & X \\
\hline Entrevistas &  &  & &  & X & X\\
\hline Questionários &  &  &  &  & X & X\\
\hline Análise Estatística &  &  & &  &  &\\
\hline Redação Capítulos Teóricos &  & X & X & X &  & \\
\hline Redação Capítulos Empíricos &  & &  &  &  & \\
\hline Conclusões/Revisão & &  & & & & \\
\hline Defesa da Tese &  &  &  & &  & \\
\hline
\end{tabular} }

\vspace{1cm}
{\scriptsize \begin{tabular}{|c|c|c|c|c|c|c|c|}
\hline \textbf{Período/} & 1\textsuperscript{\underline{o}} Trim. & 2\textsuperscript{\underline{o}} Trim. & 3\textsuperscript{\underline{o}} Trim.  & 4\textsuperscript{\underline{o}} Trim. & 1\textsuperscript{\underline{o}} Trim. & 2\textsuperscript{\underline{o}} Trim.\\ 
\textbf{Atividade} & 2017 & 2017 & 2017 & 2017 & 2018 & 2018\\ 
\hline Aulas &  &  & & & &\\
\hline Pesquisa/Monitoria no King's & X & X & X & X & & \\
\hline Análise Bibliográfica &  &  & & & & \\
\hline Modelagem Formal & X & X &  & &  &  \\
\hline Entrevistas &  &  & & &  &\\
\hline Questionários &  &  & & &  &\\
\hline Análise Estatística &  & & X & X & &\\
\hline Redação Capítulos Teóricos &  & &  &  &  & \\
\hline Redação Capítulos Empíricos & X & X & X & X & & \\
\hline Conclusões/Revisão & &  & & & X & X \\
\hline Defesa da Tese &  &  &  & &  & X\\
\hline
\end{tabular} 
}

\newpage

\bibliographystyle{apalike2}
\bibliography{bibliografia}

\end{document}
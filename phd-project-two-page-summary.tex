\documentclass[a4paper,11pt]{article}
\usepackage[utf8]{inputenc}
\usepackage{lmodern}
\usepackage[usenames,dvipsnames]{color}
\definecolor{darkblue}{rgb}{0.0, 0.0, 0.55}
\usepackage[T1]{fontenc}
\usepackage[english]{babel}
\usepackage[authoryear]{natbib}
\usepackage{babelbib} 
\usepackage{setspace}
\usepackage{amsmath}
\usepackage{amssymb}
\usepackage{graphicx}
\usepackage{microtype}
\usepackage{hyperref}
\usepackage{geometry}
 \geometry{
 a4paper,
 total={210mm,297mm},
 left=20mm,
 right=20mm,
 top=20mm,
 bottom=20mm,
 }

\exhyphenpenalty=1000
\hyphenpenalty=1000
\widowpenalty=1000
\clubpenalty=1000

\hypersetup{pdftitle={The Violent Order: Power and Governance in Brazil's Primeiro Comando da Capital}, pdfauthor={Danilo Alves Mendes Freire}, pdfsubject={PhD Research Proposal}, pdfkeywords={Brazil, Collective Action, Criminal Organisations, Prison Gangs, São Paulo},linkcolor=darkblue, citecolor=darkblue, urlcolor=darkblue, breaklinks=true, colorlinks=true}

%opening

\title{The Violent Order: Power and Governance in\\ Brazil's Primeiro Comando da Capital}
\author{Danilo Alves Mendes Freire}
\date{24th March 2015}

\begin{document}
\maketitle

\singlespacing

\noindent \textbf{Topic Area: Collective Action in Prisons}. The thesis shall examine how the Primeiro Comando da Capital (\textit{First Command of the Capital}, henceforth ``PCC''), Brazil's largest prison gang, has gained control over the country's penal system, and by what means the group promotes order and governance to Brazil's massive inmate population. The private provision of public goods has long puzzled economists and political scientists \citep{hayek1960constitution, olson1965logic, ostrom1990governing}. Whereas there are several examples of successful informal institutions \citep{helmke2004informal}, prison gangs pose a serious challenge to existing theories of collective action \citep{sheptycki2003governance, skarbek2011governance, skarbek2014social}. Due to the pervasive threat of violence, lack of interpersonal trust, and limited access to reliable information in the prison environment, inmates have strong incentives to disrupt collective arrangements and exhibit strictly selfish behaviour \citep{gambetta2009codes, liebling2012social}. Nevertheless, many prison gangs have managed to break this social dilemma and effectively establish decision-making mechanisms. Since the literature predicts that the presence of group interest alone is not a sufficient condition to collective action \citep{marx2012achtzehnte, olson1965logic}, a throughout study of the incentives and punishments employed by prison gangs is therefore required. In this regard, the research will use the PCC as a case study to address these issues.

\vspace{.45cm}

\noindent \textbf{Research Questions}. The purpose of the thesis is twofold. First, it intends to explain under what conditions the PCC has adopted different strategies -- selective or indiscriminate violence, social or monetary incentives -- to foster coordination and punish defectors. Second, the research aims to use the PCC example to derive a theoretical framework to understand prison gang choices. In order to achieve these goals, the study suggests four testable hypotheses. $H_1$: The recent growth of Brazil's inmate population has put pressure over the convicts' property rights, what has in turn reduced the costs of collective action; $H_2$: In its early stages, the PCC had access only to incomplete information and was thus forced to use indiscriminate violence, what explains the high levels of mortality in that period; $H_3$: The PCC has progressively offered a series of market incentives to its members, but only after agent-principal problems had become ubiquitous; $H_4$: PCC-controlled facilities have less prison staff violence against inmates mainly because of the gang's ability to establish \textit{non-credible} threats. Deliberate self-harm is an effective bargaining strategy in restricted environments, and it may be the optimum strategy when pursued collectively.

\vspace{.45cm}

\noindent \textbf{Significance to Knowledge}. The PCC is Brazil's and South America's most influential prison gang: it currently dominates about 90\% of the prisons in their native São Paulo state (Latin America's richest), has established itself in 22 of the 27 Brazilian states, profits about 50 million dollars per year, and has allegedly elected their own representatives in Brazil's last elections \citep{biondi2008etica, veja2013}. In this sense, the study has considerable theoretical and practical significance. Research on the PCC has been exclusively descriptive and confined to Portuguese speakers, so the thesis is notably innovative in its theory building ambitions and the dialogue it opens with the literature on collective action, state formation, and costly signals. The practical relevance of the research is also noteworthy. Since prison gangs are Brazil's most pressing security threat \citep{dias2011pulverizaccao}, a better understanding of those groups in general, and of the PCC in particular, can help the public authorities to design more effective policies against prison violence and urban crime.

\newpage

\noindent \textbf{Research Methods}. The thesis will adopt a mixed methods approach. More specifically, it will follow the ``tripartite methodology'' suggested by \citet[165]{laitin2003perestroikan}, a combination of qualitative methods, formal models, and statistical analysis. The qualitative part will be carried out first. It consists in in-depth interviews with convicts using snowball sampling \citep{goodman1961snowball}. Such interviews are intended to provide a first understanding of the inner workings of the PCC and help us approach both prison staff and convicts. Contacts will be established via the Centre for the Study of Violence -- which has already conducted several research projects about São Paulo's inmate population -- and the Secretary for Penitentiary Administration, the government body responsible for the prison facilities in the state. Although snowball sampling is a non-probabilistic technique, it is suitable for studies on criminal organisations. In environments where trust is scarce, such methods of ``chain referral'' are likely to be the best way to proceed \citep{johnston2010sampling}.

Next, I intend to conduct survey experiments with the inmates. The goal is to obtain reliable estimates about PCC support in prisons and assess what sort of incentives the gang has been using. Experiments with random assignment are the golden standard of causal inference, and although their use has grown explosively in the past years \citep{druckman2011cambridge,morton2010experimental}, they have never been applied to explain São Paulo's prison environment \citep{dias2011pulverizaccao}. While the specifics are yet to be determined, I plan to employ endorsement experiments and randomised response methods as developed in a series of paper by Kosuke Imai and his colleagues \citep{blair2014comparing, bullock2011statistical, rosenfeld2014empirical}. They represent the state of art in survey techniques to extract truthful answers to sensitive questions and can be easily analysed with Bayesian hierarchical measurement models \citep{fox2001bayesian, gelman2014bayesian,imai2011multivariate}.

Lastly, the thesis will make use of agent-based modelling (``ABMs'') to explore possible causal mechanisms to the findings of the previous stages. Since gathering information on the PCC's temporal dynamics is notably difficult, ABMs can fill this gap by providing an intuitive yet rigorous way to assess the impact of social norms and evaluate feedback effects, thus capturing patterns that cannot be fully captured by experiments or individual interviews \citep{axelrod1997complexity, epstein2006generative}. Furthermore, ABMs also allow scholars to test counterfactual (``what if?'') scenarios, what facilitates the sensitivity analyses and meets the theoretical requirements of Rubin's causal model \citep{holland1986statistics, rubin2005causal}.

\vspace{.45cm}

\noindent \textbf{Expected Outcomes}. There are three expected results from this project. First, in line with several authors who have written about emerging order under conditions of anarchy \citep{buchanan1975limits, nozick1974anarchy}, it is expected that the establishment of the PCC as a major gang follows a similar path to that of state formation. That is, the players understand that the best market structure for violence is that of monopoly, since the monopolist have incentives to ``underproduce'' violence and allocate the remaining resources elsewhere. However, the incentive for a prison monopolist to emerge should be taken as given, as it only occurs after the a substantial rise in costs for property right protection. This reflects Brazil's prison situation in the 1990s. Second, contra strict rational choice theories, it is expected that prisoners can show altruistic behaviour in response to social norms \citep{skarbek2012prison}. As prisons become overcrowded and evolve into complex systems, it is expected that inmates differentiate themselves into three ideal types of actors, \textit{rational egoists}, \textit{conditional cooperators}, and \textit{and willing punishers/rewarders} \citep{ostrom2000collective}. Finally, it is also expected that informers play a major, albeit underappreciated, role in lowering the violence perpetrated by a prison gang. The hypotheses suggested above indicate a strong association between information access and the use of physical violence, with the main causal mechanism being the ability to use selective punishment via informants. Hence, informants may be the key to an efficient prison organisation.

\vspace{.45cm}

\noindent \textbf{Resources Required}. The research presented in this project can be conducted with minimal resources. All computational models and questionnaires will be analysed with free, open source tools (\texttt{Stan} and \texttt{NetLogo}), thus no licensing fees will be required. No language training or research assistants will be needed as I am a Portuguese native speaker and have already conducted research on the gang (Master's Degree). Accommodation and travelling can be easily and cheaply provided: not only the Brazilian Real is currently devalued, but Brazil has a good highway network and several air careers with low fares. Additionally, I have already contacted the Centre for the Study of Violence and the Department of Political Science at the University of São Paulo, which have offered logistic support at no cost.

\newpage
\bibliographystyle{apalike2}
\bibliography{bib-two-pages}
\end{document}
